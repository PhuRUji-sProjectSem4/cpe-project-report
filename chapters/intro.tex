\chapter{\ifenglish Introduction\else บทนำ\fi}

\section{\ifenglish Project rationale\else ที่มาของโครงงาน\fi}
เนื่องจากการแข่งขัน E-sport กำลังเติบโตเป็นอย่างมากและเป็นที่นิยมขึ้นเรื่อยๆ ในการจัดการแข่งขันทัวร์นาเมนต์นั้น
ต้องใช้หลายแอพพลิเคชันในการจัดการทั้งใช้ในการจัดการคน การทำตารางแข่ง การเก็บบันทึกผล และการแสดงผลลัพธ์ ซึ่งต้องใช้
แอพพลิเคชันหลายอัน เช่น Excel, Photoshop และ Google Dirve   จึงเป็นที่มาของโครงงานนี้ที่จะนำสิ่งต่างๆที่จำเป็นมาไว้ในเว็ปไซต์นี้เว็ปไซต์เดียว
และ อีกหนึ่งเหตุผลก็คือต้องการยกระดับวงการ E-sport และให้โอกาศแก่ผู้ที่ต้องการเริ่มต้นเข้าสู่วงการที่ไม่ได้มีทุนมากมาย แต่มีใจที่จะมาสายงานนี้  


\section{\ifenglish Objectives\else วัตถุประสงค์ของโครงงาน\fi}
\begin{enumerate}
    \item พัฒนาเว็บแอพพลิเคชันที่ใช้ในการจัดสร้าง และจัดการทัวร์นาเมนต์ ได้ในที่เดียว
    \item ใช้เป็นที่เก็บผลงานของเหล่านักกีฬา E-sport ได้
    \item สามารถนำไปใช้ต่อยอดได้ในอนาคต
\end{enumerate}

\section{\ifenglish Project scope\else ขอบเขตของโครงงาน\fi}

\subsection{\ifenglish Software scope\else ขอบเขตด้านซอฟต์แวร์\fi}
\begin{enumerate}
    \item การสร้างทัวร์นาเมนต์จะมีแค่แบบเดียวคือแบบ Single Knock Out 
    \item สามรถจัดเก็บประวัติ และจัดการโปรไฟล์ได้ตามที่มีกำหนดให้เท่านั้น
    \item การกำหนดผลการแข่งขั้นนั้นขึ้นอยู่กับผู้จัดการแข่งขันเป็นผู้จัดการไม่สามรถตรวจสอบเพื่อยืนยันได้
    \item ระบบความปลอดภัยขอเว็ปไซต์ใช้เพียงแค่ Token เท่านั้น
\end{enumerate}

\section{\ifenglish Expected outcomes\else ประโยชน์ที่ได้รับ\fi}
\begin{enumerate}
    \item ช่วยอำนวยความสะดวกให้ผู้ดูแลการแข่งขัน
    \item เป็นที่เก็บผลงานประวัติการแข่งขัน
    \item สามารถนำไปใช้ร่วมกับโปรเจค StartUp ของทีมผมเพื่อสร้าง Ecosystem ของวงการ E-sport
\end{enumerate}

\section{\ifenglish Technology and tools\else เทคโนโลยีและเครื่องมือที่ใช้\fi}

\subsection{\ifenglish Software technology\else เทคโนโลยีด้านซอฟต์แวร์\fi}
\begin{enumerate}
    \item React[1] -
    \item JavaScript[2] -
    \item Nest[3] -
    \item TypeScript[4] -
    \item PostgreSQL[5] - 
    \item Firebase Cloud Storage[6] -
    \item Docker[7] -
    \item Heroku[8] -
\end{enumerate}

\section{\ifenglish Project plan\else แผนการดำเนินงาน\fi}

\begin{plan}{9}{2022}{2}{2023}
    \planitem{9}{2022}{10}{2022}{ออกแบบUX/UI เลือกเครื่องมือ และเขียนรายงาน}
    \planitem{10}{2022}{11}{2022}{พัฒนาฐานข้อมูล}
    \planitem{11}{2022}{2}{2023}{พัฒนาเว็ปแอพพลิเคชัน}
    \planitem{11}{2022}{2}{2023}{ทดสอบ}
    \planitem{2}{2023}{2}{2023}{Deploy web hosting}
    \planitem{2}{2023}{2}{2023}{จัดเตรียมนำเสนอ และสรุปผล}
\end{plan}

\section{\ifenglish%
Impacts of this project on society, health, safety, legal, and cultural issues
\else%
ผลกระทบด้านสังคม สุขภาพ ความปลอดภัย กฎหมาย และวัฒนธรรม
\fi}

ในการทำโครงงานนี้คาดว่าจะช่วยลดขั้นตอน และความยุ่งยากในการจัดสร้าง และจัดการทัวร์นาเมนต์
มากไปกว่านั้นเว็ปไซต์นี้จะเป็นศูนร่วมของเหล่านักกีฬา E-sport และเหล่าผู้จัดงาน ในอานาคต
