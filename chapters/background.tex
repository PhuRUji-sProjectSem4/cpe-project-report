\chapter{\ifenglish Background Knowledge and Theory\else ทฤษฎีที่เกี่ยวข้อง\fi}

การทำโครงงาน เริ่มต้นด้วยการศึกษาค้นคว้า ทฤษฎีที่เกี่ยวข้อง หรือ งานวิจัย/โครงงาน ที่เคยมีผู้นำเสนอไว้แล้ว ซึ่งเนื้อหาในบทนี้ก็จะเกี่ยวกับการอธิบายถึงสิ่งที่เกี่ยวข้องกับโครงงาน เพื่อให้ผู้อ่านเข้าใจเนื้อหาในบทถัดๆ ไปได้ง่ายขึ้น


\section{UX Design}
UX Design(User Experience Design) คือ การออกแบบเพื่อให้ผู้ใช้งานมีประสบการณ์ในการใช้งานที่ดีในเว็ปไซต์ หรือแอพพลิเคชันนั้นๆ
ไม่ว่าจะเป็นความง่ายในการใช้งาน ความต่อเนื่องในการใช้งาน การสื่อความหมาย ฯลฯ ตาม UX Law[9] เพื่อประสบการณ์การที่ดีที่สุดของผู้ใช้งาน


\section{\ifenglish%
\ifcpe CPE \else ISNE \fi knowledge used, applied, or integrated in this project
\else%
ความรู้ตามหลักสูตรซึ่งถูกนำมาใช้หรือบูรณาการในโครงงาน
\fi
}

\begin{enumerate}
    \item Database: การออกแบบดาต้าเบส
    \item IT Infra and Cloud Tech: การใช้ Docker ในการทำ Docker Image และ Container เพื่อในไปใช้ Deploy
\end{enumerate}

\section{\ifenglish%
Extracurricular knowledge used, applied, or integrated in this project
\else%
ความรู้นอกหลักสูตรซึ่งถูกนำมาใช้หรือบูรณาการในโครงงาน
\fi
}

\begin{enumerate}
    \item ภาษาโปรแกรมมิ่งสำหรับทำเว็ปแอพพลิเคชัน และ Framework ที่มีประสิทธิภาพในการใช้งานเพื่อเพิ่มความสะดวก และรวดเร็วในการพัฒนาเว็ปแอพพลิเคชั่น
    \item การจัดเก็บรูปภาพ และการดึงมาใช้ร่วมกับการใช้ฐานข้อมูลแบบSQL โดยการใช้ Cloud มาเป็นตัวช่วย
\end{enumerate}
